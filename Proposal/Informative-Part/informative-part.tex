\chapter{Informative Part}
\section{Team}
\begin{itemize}
  \item Fabio J. Matos Nieves
  \item Héctor A. Rivera Hernández
  \item Sergio A. Meléndez Padilla
  \item Alexander J. Gonzalez Suarez
  \item Josean Rodriguez Lugo
\end{itemize}
\section{Current Situation, Needs, Ideas}
\subsection{Current Situation}
\hspace{1cm} The second hand shoe market being projected to have a market size of \$893.9 million \cite{SecondHandDesigner}, with expected growth at a steady rate of 7.8\% \cite{SecondHandDesigner2022}, is it safe to assume that this market is not going away any time soon. Being a booming second hand market, there will be malicious actors within. Mainly that the shoe industry is the most affected my counterfeit products \cite{InfographicIndustriesMost2019}. Furthermore, since it is a booming industry, counterfeiting is on the rise and knowledge of identifying counterfeits are scarce and ever evolving. This is further compounded by the fact that the barrier to entry for this industry is quite high, so resellers of legitimate shoes are financially discouraged in making their product available and thus making it seem that counterfeit products are more prevalent then they actually are. In the continental USA, there are platforms like GOAT.com that are a market place for the reselling of shoes but none exist for Puerto Rico with a focus on independent verification and validation of the legitimacy of the products being sold on the platform.
\subsection{Need}
\hspace{1cm} There currently is a need for a platform that can openly host resellers of limited edition shoes, whilst providing transparent verification of each and every product hosted on the platform. Since there is a large proportion of counterfeit products within the second hand shoe market, an open repository of information about counterfeit products is a must in order subside the purchase of counterfeit products on all platforms. At the time of this proposal, there is no such platform that meets all of these criteria. Therefore, there is a need to develop a:
\begin{itemize}
  \item Domain Description
  \item Requirements Prescription
  \item Software Design
  \item Implementation
\end{itemize}
for a platform that can meet all of the aforementioned criteria.
\subsection{Ideas}
Our idea for helping alleviating this problem is creating an online platform where buyers and sellers alike can have transparency and honesty throughout the entire shopping process. The big distinguishing feature of this platform is the recording of the review process of the product in order guarantee authenticity of the product on sale.
\section{Scope, Span, and Synopsis}
\subsection{Scope and Span}
Scope:
\begin{itemize}
  \item The domain for this project is the second hand shoe market.
  \item This platform will provide the ability to search, view and buy shoes and showcase the review process on every listing on the platform.
  \item As for the implementation for this platform it will be a Web Application that will function on Chromium based browsers, Firefox and Safari and both desktop and mobile platforms.
  \item This project will also delve into domain engineering, requirements engineering, software architecture, interface requirements, machine requirements, events, actions and behaviors, required for this project.
\end{itemize}
Span:
\begin{itemize}
  \item Time Frame: 8 Weeks and 6 Days, to submit on May 3, 2023.
        \begin{itemize}
          \item Front End: Bootstrap and Jinja \cite{JinjaJinjaDocumentation}
          \item Back End: Flask\cite{WelcomeFlaskFlask}
          \item Architecture: REST API
        \end{itemize}
  \item Agile methodology:
        \begin{itemize}
          \item Creating the home page.
          \item Creating the log in portal.
          \item Automating the creation of new pages for newly added products.
          \item Implement search by brand, name, date added and price.
          \item Connect front end with the database.
          \item Optimize the addition of files to the database.
        \end{itemize}
  \item Deliverable:
        \begin{itemize}
          \item The deployment of the project via a web app.
          \item 1 progress report and 1 final report.
          \item Possible feature additions and bug fixes upon deployment.
        \end{itemize}
\end{itemize}
\subsection{Synopsis}
This web application that will be developed in roughly 8 weeks consisting of 12 sprints of 4 days each, it will consist of a shoe selling platform in which people will be able to buy legitimate shoes in a secure way without getting scammed and sellers could promote their business through the platform as well.
\section{Other Activities Than Just Developing Source Code}
\begin{itemize}
  \item Creating a rough sketch of the domain.
  \item Simulating the domain in a tool like alloy.
  \item Define hardware requirements.
  \item Defining terminology
  \item Documenting Source Code.
  \item Creating a narrative.
  \item Defining events, actions and behaviors.
  \item Defining interface and feature requirements.
  \item Validating and verifying the project feature set.
\end{itemize}
\section{Derived Goals}
\begin{itemize}
  \item Familiarize the group in web development using python.
  \item Inform the stakeholders on how to recognize legitimate products.
  \item Foment a community of openness and transparency within the second hand shoe market.
\end{itemize}
\section{Log Book}
\begin{itemize}
  \item \hlc{green}{Added Personas.}
  \item \hlc{green}{Added Relevant Stakeholders.}
  \item \hlc{green}{Added Business Process for Relevant Stakeholders}
\end{itemize}
\section{Time management, scheduling, planning}
For one we first set out on creating the login/ sign up system for the web app. When we first set out planning for this feature to be implemented, we set the sprint target to be the implementation of the login/sign up system. However, since this feature encompasses the whole stack, front end to back end, we decided to break up the feature sprint into smaller sprints. The smaller sprints were as follow:
\begin{itemize}
  \item Home Page Demo
  \item Login Form
  \item Sign up Form
  \item SQLITE Database creation
  \item Flask integration
\end{itemize}
For example, since the Home Page Demo was done first by Fabio, he estimated that an initial home page concept design would take 6 hours to create, since it was his first time using the HTML/ and Bootstrap in a project. In reality it took around 3 hours of work to implement a concept design. Then, after a group brainstorming session, it was decided that the home page concept needed a redesign, and Fabio estimated that a redesign up take 2 hours, in reality it took 2 hours and 15 minutes.
\\The back end was implemented by Hector, and he estimated that setting up the back end, creating the routes, views, models of the database and the sign up/log in feature would take 4 hours and 30 minutes to create, since it was his first time using Python Flask, as well setting up a database. In reality it took 3 hours of work to implement the back end and the login/ signup feature.
\section{Agile}
As explained above for our first feature implementation, we decided that the first feature oriented sprint would be implementing the log in/ sign up system. However, since the log in/sign up system encompasses the whole web stack, a lot of ground work needed to be implemented first, thus we decided split up this sprint into 5 smaller sprint to fully implement this feature, these were:
\begin{itemize}
  \item Home Page Demo $\boxtimes$
  \item Login Form $\boxtimes$
  \item Sign up Form $\boxtimes$
  \item SQLITE Database creation $\boxtimes$
  \item Flask integration $\boxtimes$
\end{itemize}
As of now, we have the following sprint backlog:
\begin{itemize}
  \item Login System $\boxtimes$
  \item Product Page system $\square$
  \item Recent shoe entry preview system $\square$
  \item Profile system (Optional) $\square$
  \item Search system (Optional) $\square$
\end{itemize}
