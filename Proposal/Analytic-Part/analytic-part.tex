\chapter{Analytic Part}
\section{Concept Analysis}
\hspace{1cm} To begin with, in the “Domain Rough Sketch", 
"sneaker head" represents a market of interest which has “secondhand sellers” who sell “limited edition shoes” to the “customer.” 
The current problem is that Puerto Rico doesn't have a platform that specializes in this. Furthermore, after careful investigation it was also found that big reselling platforms don't have a transparent verification system. Finally, by having a platform like this solves the doubts of getting a "knock-off" version of someone's favorite limited edition shoe and opens a brand new market in Puerto Rico.
\section{Validation and Verification}
\hspace{1cm} The validation procedures ensure that the sneakers meet the needs of authenticity. This includes evaluating the feasibility, applicability and effectiveness of the proposed corrective actions.
Market research is conducted to better understand the needs, preferences and behaviors of the target market related to all the sneakers. Market trend analysis, customer reviews, and competitive research all fall into this category.
Seller verification is being taken account by rigorously verifying the legitimacy of the merchant and the sneakers they sell. This includes verifying the identity of the seller.
User acceptance testing is conducted to ensure that the website meet the needs of the user. Verification of said features will be verified using feedback from domain experts and stakeholders by some of the following methods:
\begin{itemize}
  \item A/B testing
  \item Requirements Testing
  \item Domain Engineering testing
  \item Interface Engineering testing
\end{itemize}
Any addition to the project will be analyzed by the team members to determine if it’s aligned with the primary needs of the stakeholders. During the development phase, the team will have open discussions to ensure that the system is developed by the best practices. Buyers must see an Enhanced experience with the process of buying secondhand sneakers by the elimination of counterfeit products.
\section{Requirements}
\subsection{Domain Requirements}
\begin{itemize}
  \item \hlc{green}{A shoe can only have 1 owner at a time.\\
        Since articles of clothing are material and finite items, they can only have 1 person be the owner. As such A shoe cannot be listed as two separate entries on the store neither can the same shoe be sold by 2 more different sellers on the store.}
  \item \hlc{green}{Money associated to a sale can to point to point.\\
        This means that transactions within the store cannot come from nowhere. For example, a customer buys a shoe and the money is sent to SNKRS Trust to take a service fees and then the remainder is sent to the seller. This transaction cannot go from the company, to the buyer and then to the seller or from the seller to the buyer to the company. There is a set path how transactions occur.}
  \item \hlc{green}{There cannot a sale on the SNKRS Trust if there is no entry of the product.\\
        For a product to be listed on the site, as mentioned in the business process section, it first has to be verified. Once the product is successfully validated, it is listed on the website.}
  \item \hlc{green}{SNKRS Trust verification process is only concerned with shoes listed on the site.\\
        Any shoe that is purchased outside the site is not guaranteed to be independently verified that it is legitimate.}
\end{itemize}
\subsection{Interface Requirements}
\subsection{Machine Requirements}
\section{Software Design}
