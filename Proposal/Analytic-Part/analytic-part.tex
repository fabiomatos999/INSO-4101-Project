\chapter{Analytic Part}
\newpage
\section{Concept Analysis}
\hspace{1cm} To begin with, in the “Domain Rough Sketch", 
"sneaker head" represents a market of interest which has “secondhand sellers” who sell “limited edition shoes” to the “customer.” 
Within the requirements a “database” within the “system-to-be” stores the sellers, users, and shoes information. 
Furthermore, the “system-to-be” compiles a list of shoes based on their unique brands and models. 
This concept allows “users” to purchase an item within a list, marks them as sold and blacklist items.
Finally, the “system-to-be” ensures that the “users” get a guaranteed legitimate limited edition shoe.
\section{Validation and Verification}
\hspace{1cm} Validation and verification will be carried out throughout the Spring semester of 2022-2023, validation will be conducted by stakeholders of the project. In this project, the validation procedure will involve presenting the team’s understanding of the domain to the stakeholders/domain experts and receive feedback specifically to that. The goal of the validation is to improve the team understanding of the domain. Having a better understanding will ensure that the system is developed in a way that meets the stakeholders needs. To achieve stakeholder requirements, the team will work closely with the domain experts who can be end-users that would be buyers and/or sellers in the system. Furthermore, the development team will provide constant feedback in phases of designing and implementation of new features. When implementing new features, feedback from domain experts will be taken into consideration. Any addition to the project will be analyzed by the team members to determine if it’s aligned with the primary needs of the stakeholders. During the development phase, the team will have open discussions to ensure that the system is developed by the best practices. Buyers must see an Enhanced experience with the process of buying secondhand sneakers by the elimination of counterfeit products.
