\chapter{Analytic Part}
\newpage
\section{Concept Analysis}
\hspace{1cm} To begin with, in the “Domain Rough Sketch", 
"sneaker head" represents a market of interest which has “secondhand sellers” who sell “limited edition shoes” to the “customer.” 
Within the requirements a “database” within the “system-to-be” stores the sellers, users, and shoes information. 
Furthermore, the “system-to-be” compiles a list of shoes based on their unique brands and models. 
This concept allows “users” to purchase an item within a list, marks them as sold and blacklist items.
Finally, the “system-to-be” ensures that the “users” get a guaranteed legitimate limited edition shoe.
\section{Validation and Verification}
\hspace{1cm} Validation and verification will be carried out throughout the Spring semester of 2022-2023, 
validation will be conducted by stakeholders of the project. To guarantee a successful validation, the 
team members will be acknowledging feedback from the domain expert. A domain expert can be an
end-user that would be a buyer and or seller in the system. Furthermore, the development team will 
provide constant feedback when implementing new features. Any addition to the project will be 
analyzed by the team members to determine if it’s aligned with the primary goal of the project. During 
the development phase, the team will have open discussions to ensure that the system is developed following
best practices. Buyers must see an enhanced experience with the process of buying secondhand
sneakers by the elimination of counterfeit products. 
Various verifications methods will be implemented through the Spring semester of 2022-2023. The team 
members will develop:
Unit Testing: Individuals component of the system will be tested to ensure the proper functionality; for 
example, when a user wants to process an order, the process must complete successfully by properly 
storing the order information.
Integration Testing: The whole functionality of the website with multiple components will be tested. For 
example, when an order is completed, available quantity of the products must be modified. 
Furthermore, user must be able to access their order history in their profile.
Load Testing: When implemented, the system will be tested by having simultaneously multiple requests, 
such as creating new orders, adding to wishlist, searching for new products.
Acceptance testing: After load testing, the team will test the websites overall functionality to ensure
that it meets the system goals and expectations of stakeholders. 
\section{ER Concept Analysis Diagram}
\hspace{1cm} The ER diagram for the sneakers shopping system consist of tables such as User, Shoes, Order, 
OrderHas, Brand, Model, Cart and Wishlist. The relationship and schema should allow for efficient 
querying and retrieval of data. When implementing the system, it will be important to ensure that the 
database design aligns with the ER diagram to ensure proper storage. The system will communicate with 
the database to display and store information on the platform.

\createfigure{../Analytic-Part/Figures/ER Diagram.jpg}{ER Diagram - Sneaker Shopping System}
